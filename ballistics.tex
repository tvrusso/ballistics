\documentstyle[12pt]{article}
\begin{document}
\author{Thomas V. Russo\\A nerdy shooter with too much time on his hands}
\title{Simple ballistics calculations for nerdy shooters with not
enough time on their hands}
\maketitle
\begin{abstract}
I will attempt to summarize the basic theory behind the calculation of
simple ballistic tables from muzzle velocity, ballistic coefficient
and so forth.

This document was begun some time in 1994, when I had too much time on
my hands.  I stopped, having hardly written anything, some time in
1994, and haven't had enough time on my hands since.  I've not deleted
the file, but it is pretty useless.
\end{abstract}
\section{Statement of the problem}
In what follows I will ignore the finer points of the problem, since
they are unnecessary for the purposes of the average shooter.  The
actual trajectory of a projectile are much more complicated than I
will make them out to be.  

I will treat a bullet as a point particle moving under the influence
of two forces: gravity and drag due to the atmosphere. Wind may be
treated as a correction to this theory.  The drag will be treated
using an empirical model.

Consider a particle of mass m moving in the fired with velocity $v_0$.
Gravity is acting in the negative $y$ direction, positive $x$ is
towards the target, and the velocity is in the $xy$-plane at an angle
of $\theta$ with respect to the horizontal .  Applying Newton's third
law of motion one gets the differential equations
\begin{eqnarray}
\label{eq:newton-3}
m\ddot{x} &=& -D\cos\theta =-D \dot{x}/v\\
m\ddot{y} &=& -g - D\sin\theta = -g - D \dot{y}/v 
\end{eqnarray}
where $D$ is the drag force.  Since the initial velocity is taken
to be in the $xy$-plane for this simple treatment, I will ignore the
motion in the $z$ direction until we need to consider windage.

\section{Models for the drag force}
Consider that drag function D to be dependent on the density of the
air, the diameter of the projectile, the velocity of the projectile
and the speed of sound in some unspecified way:
\begin{equation} D=D(\rho,d,v,v_s) \end{equation}
The dependence of D on the various quantities can be determined from
{\em dimensional analysis} in the following manner.

Rescale the $x$ and $y$ variables by some parameter $L$, mass by $M$,
and the time $t$ by a quantity $T$.  Then $x\rightarrow(x/L)=x_1$,
$y\rightarrow(y/L)=y_1$, $m\rightarrow(m/M)=m_1$ and
$t\rightarrow(t/T)=t_1$.  The $x$ equation then becomes
\begin{equation}
m_1x_1^{\prime\prime}=-D(\rho_1,d_1,v_1,a_1)x_1^{\prime}/v
\end{equation}
or
\begin{equation}
\left(\frac{ML}{T^2}\right)m\ddot{x}=-D\left[ \left(\frac {M}{L^3}\right)\rho,Ld,\left(\frac
LT\right)v,\left(\frac LT\right)a\right] \dot{x}/v
\end{equation}
which therefore implies, from Equation~\ref{eq:newton-3} that
\begin{equation}
D\left[\left(\frac {M}{L^3}\right)\rho,Ld,\left(\frac LT\right)v,\left(\frac LT\right)a\right]=\left(\frac
{ML}{T^2}\right) D(\rho,d,v,a)
\end{equation}
Since $L$, $M$, and $T$ were arbitrary constants, we may set $L=\frac
1d$, $M=\frac{1}{\rho d^3}$ and $T=\frac vd$ to obtain
\begin{equation}
D(1,1,1,(a/v))=\frac{1}{\rho d^2v^2}D(\rho,d,v,a)
\end{equation}
or
\begin{equation}
D(\rho,d,v,a)=\rho d^2v^2K_D(v/a)
\end{equation}
That is, the drag force depends on the atmospheric density, the
velocity of the projectile, its diameter and a second function, $K_D$,
called the {\em drag coefficient}, which depends only on $v/a$.  The
ratio $v/a$ is known as the {\em mach ratio}.

The expressions above may be manipulated down to  
\begin{eqnarray}
\ddot{x}&=&\frac{\rho d^2 v^2 \cos \theta}{m} K_D(v/a)\\
        &=& \frac{\rho d^2 v K_D(v/a)}{m} \dot{x} \\
        &=& \left[\frac {\rho}{\rho_0}\right] \left[ \rho_0 v K_D(v/a)
\right] \frac{d^2}{m} \dot{x}
\end{eqnarray}
The first bracketed expression is the ratio of the atmospheric density
to some reference density $\rho_0$. This ratio is often referred to as
$H$.  The second expression is known as $G(v,a,\rho_0)$, and the
fraction $C=\frac{m}{d^2}$ is known as the {\em ballistic coefficient}
for the specified drag function $G$.

The drag functions $G$ are not generally known in analytical form.
The best that can be had are empirical tabulations of drag functions
for specific projectile shapes.  These are generally tabulated as
$G(v)=G(v,a_0,\rho_0)=\rho_0 v K_D(v/a_0)$, where $a_0$ and $\rho_0$
are the speed of sound and atmospheric density at a standard
temperature and pressure.  The speed of sound at any other temperature
is given by $a=a_0 (T/T_0)^{1/2}$, so the function $G(v,a,\rho)$ can
be obtained using the tables and the relation
$$G(v,a,\rho)=\frac{\rho}{\rho_0}\left(\frac{T}{T_0}\right)^{1/2}
G(v\left(\frac{T_0}{T}\right)^{1/2}).$$

Unfortunately the drag force $D=HG/C$ is strictly correct only when
the projectile being studied is precisely the same shape as the one
for which $G$ was tabulated.  It is often possible, however, to use
the same expression by modifying the ballistic coefficient to account
for the differing shape.  This is done by introducing a {\em form
factor}, $i$, so that the ballistic coefficient is given by
$C=\frac{m}{id^2}$.

Commonly used drag functions are the Ingall's table, the British
1909 table, the Gavr\'e table, and the G1 table.  The G7 table is also
used for very low drag (VLD) bullets.  Each table is appropriate for
bullets of a particular range of shapes, and it is essential to use a
ballistic coefficient appropriate to the table you're using.  The
ballistic coefficients listed by bullet manufacturers are most likely
to be derived for the G1 drag model.

\section{Approximations needed to calculate trajectories}

\end{document}
